\documentclass[11pt]{article}
% Paquetes
\usepackage{amssymb}
\usepackage{amsmath,lipsum}
\usepackage[most]{tcolorbox}
\usepackage{setspace}
\usepackage{svg}
\usepackage{ifluatex}
\usepackage[spanish]{babel}
\usepackage[shortlabels]{enumitem}

\ifluatex
  \usepackage{pdftexcmds}
  \makeatletter
  \let\pdfstrcmp\pdf@strcmp
  \let\pdffilemoddate\pdf@filemoddate
  \makeatother
\fi

\newcommand{\incsvg}{\includesvg}
%% Ajustes del documento
\setlength{\parindent}{0em}
\setlength{\parskip}{1em}
\setlength{\fboxsep}{1em}
\renewcommand{\baselinestretch}{1.2} 
\title{Sistema dinámicos I}
\author{Basado en las lecturas de João Pedro Morais\\
    Notas tomadas por Daniel Vélez Moyado}
\date{Primavera 2026}

\begin{document}
\maketitle 
\pagebreak
\begin{spacing}{0.1}
\tableofcontents
\end{spacing}
\pagebreak




\section{Ecuaciones diferenciales de primer grado}
La idea principal de esta materia gira alrededor de la idea 
de resolver ecuaciones diferenciales y analizar distintas propiedades de ellas.

\fbox{\begin{minipage}{\dimexpr\textwidth-2\fboxsep-2\fboxrule\relax}
\paragraph{Definición} Una \textbf{ecuación diferencial} es aquella que involucra una función ($y$),
sus variables ($x$) y derivadas de dicha función ($y^{(k)}$). Su forma general es
\begin{equation*}
   F(x, y, y', y'', ..., y^{(n)}) = 0, \quad y := y(x) \quad y^{(k)} := \frac{d^k y}{dx^k}, \quad k \in \mathbb{N}
\end{equation*}
\end{minipage}}

\textbf{[1.1] Ejemplo:} Suponga la siguiente ecuación $y' = -k y$ para $y(x) > 0, \forall x \in \mathbb{R}$. La 
metodología para resolver estas ecuaciones es pasar todos los terminos de la función de un lado y dejar todo lo demás
del otro lado de la ecuación, tomando el ejemplo llegaríamos a
\begin{equation*}
   \frac{y'}{y} = -k \iff \int \frac{y'}{y} dx = \int -k  dx + c, c \in \mathbb{R}
\end{equation*}
Considerando que $y' =  \frac{d y}{ dx} \iff y' dx = dy$
\begin{equation*}
   \int \frac{y'}{y} dx = \int -k  dx + c \iff  \int \frac{1}{y} dy = \int -k  dx  \iff \ln|y| = -kx  + c
\end{equation*}
Dejar la ecuación de esta manera significa obtener una \textbf{solución implícita} ecuación diferenical, 
\begin{equation*}
   \ln|y| = -kx  + c \iff |y| = e^{-kx  + c} \iff y = \pm e^{-kx  + c} \iff y = Ae^{-kx}
\end{equation*}
Tal que $A \in \mathbb{R}_{\neq 0}$. La expresión $ y = Ae^{-kx}$ sería la solución explícita de la ecuación diferencial. Sin embargo 
pese que llegamos a que $A\neq 0$, esto no excluye la posibilidad (de manera general) de que $y(x) = 0$ sea una solución. Para saber si la solución \textbf{trivial} $y(x) = 0$
es una solución hay que checar las condiciones sobre las cuales se plantea el problema y el desarrollo de toda la ecuación. En este caso tenemos $y >0$ por lo que la solución trivial no forma
parte del conjunto de soluciones. 

Podemos observar que existen un conjunto de soluciones a la ecuación diferencial, ya determinamos que
este se verá de la siguiente forma: 
\begin{equation*}
   \{ y : y = Ae^{-kx}, A \in \mathbb{R}_{\neq 0}  \}
\end{equation*}
Sin embargo, si el problema presenta condiciones iniciales, podemos determianr una solución particular.

\fbox{\begin{minipage}{\dimexpr\textwidth-2\fboxsep-2\fboxrule\relax}
\paragraph{Definición} Una \textbf{solución particular} es aquella solución $y : \mathbb{R} \to \mathbb{R}$
que satisface $y(x_o) = \alpha, \alpha \in \mathbb{R}$ para alguna $x_0 \in \mathbb{R}$. Se define $y_0 := y(x_0)$
\end{minipage}}

Retomando el ejemplo \textbf{[1.1]}, si agregamos la condición inicial $(x_0, y_0)$. Si queremos 
encontrar la solución partiuclar sólo basta resolver $y_0 = A e^{-k x_0}$.

\fbox{\begin{minipage}{\dimexpr\textwidth-2\fboxsep-2\fboxrule\relax}
\paragraph{Teorema} El \textbf{intervalo máximo de existencia}, denotado $I_M$ de una solución
particular $y$ es el mayor sub-intervalo del dominio de $y$ que:
\begin{enumerate}[a., topsep=0pt, itemsep=-2px]
\item $x_0 \in I_M$
\item $y \in C^1[I_M]$ (La solución es de clase $C^1$, lo mismo a decir que $y'$ es continua en $I_M$)
\end{enumerate}
\end{minipage}}
\subsection{Ecuaciones diferenciales de variables separables}
\fbox{\begin{minipage}{\dimexpr\textwidth-2\fboxsep-2\fboxrule\relax}
\paragraph{Definición} Una ecuación diferencial $y'(x) = g(x,y)$ se dice que es de variables separabes es:
\begin{equation*}
    g(x,y) = M(x)N(y) \Rightarrow y'(x) =   M(x)N(y)
\end{equation*}
Tal que $N(y) \neq 0$
\end{minipage}}

La manera de resolver este tipo de ecauciones son como mostró en el ejemplo \textbf{[1.1]}.
De manera general debemos de resolver la siguiente integral
\begin{equation*}
   \int \frac{1}{N(y)}dy = \int M(x) dx + x, c \in \mathbb{R}
\end{equation*}

\subsection{Ecuaciones diferenciales homogéneas}
\fbox{\begin{minipage}{\dimexpr\textwidth-2\fboxsep-2\fboxrule\relax}
\paragraph{Definición} Una ecuación diferencial de primer orden $y'(x) = g(x,y)$ se dice que es  
\textbf{homogénea} (de grado 0) cuando:
\begin{equation*}
   g(\lambda x, \lambda y) = \lambda^0 g(x,y) = g(x,y)
\end{equation*}
\end{minipage}}

Para resolver este tipo de ecuaciones debemos de considerar que podemos escribir a $y$ como 
$y(x) = v(x)  x$ para alguna función $v : \mathbb{R} \to \mathbb{R}$. Entonces
\begin{equation*}
   y' = g(x,y) = g(x, v(x) x) = g(1, v(x))
\end{equation*}
A su vez también sabemos que $y' = v'(x)x + v(x)$, considerando $v := v(x)$, se sigue:
\begin{equation*}
   g(1, v) = v'x + v \iff v'x =  g(1, v) - v  \iff  \frac{v'}{g(1, v) - v } = \frac{1}{x}  
\end{equation*}
Notemos que estas implicaciones son verdaderas sii $x \neq 0$ y $g(1,v) - v \neq 0$. 
Finalmente se integran ambos lados para obtener $v(x)$, una vez teniendo dicha función sólo 
bastaría multiplicar todo por $x$ para obtener la solución $y(x)$.


\subsection{Ecuaciones diferenciales exactas}

\fbox{\begin{minipage}{\dimexpr\textwidth-2\fboxsep-2\fboxrule\relax}
\paragraph{Definición} Un \textbf{campo vectorial} sobre un conjunto de $\mathbb{R}^n$ es una 
función con valores vectoriales: $F : D \to \mathbb{R}^n$, se dice que $F$ es un campo vectorial $c^k$ 
si como función es $k$ veces diferenciable con continuidad en $D$.
\end{minipage}}

\fbox{\begin{minipage}{\dimexpr\textwidth-2\fboxsep-2\fboxrule\relax}
\paragraph{Definición} Un campo vectorial $(c^k, F)$ sobre $D \subseteq \mathbb{R}^n$ se llama \textbf{campo gradiente}
si existe una función $f : D \to \mathbb{R}$ en $c^{k+1}$ tal que 
\begin{equation*}
   F = \nabla f = \left( \frac{\partial f}{\partial x_1}, ... , \frac{\partial f}{\partial x_n} \right)
\end{equation*}
\end{minipage}}

\fbox{\begin{minipage}{\dimexpr\textwidth-2\fboxsep-2\fboxrule\relax}
\paragraph{Definición} Se dice que ecuación diferencial de primer orden
\begin{equation*}
   M(x,y) + N(x,y) y' = 0
\end{equation*}
con $M,N : D \subseteq \mathbb{R}^2 \to \mathbb{R}$ es \textbf{diferencial exacta} siempre 
que el campo vectorial $(M,N)$ sea un campo gradiente en $D$. En otras palabras que exista $\phi : D \to \mathbb{R}$
tal que $\nabla \phi = (M, N)$.
\end{minipage}}

La importancia de la función $\phi$, tal que  $\nabla \phi = (M, N)$, es que para resolver $M(x,y) + N(x,y) y' = 0$
podemos realizar la siguiente sustitución a la ecuación :
\begin{equation*}
    \frac{\partial\phi}{\partial x} + \frac{\partial \phi}{\partial y}y' = 0 \iff \frac{d}{ d x}[ \phi (x, y(x))] = 0 \iff \phi(x, y(x)) = k, k \in \mathbb{R}
\end{equation*}

Como la función $\phi$ es una constante, la solución implícita a una ecuación diferencial exacta es $\phi(x, y(x)) - k = 0$. El reto en resolver este tipo de 
ecuaciones  radica en encontrar la función $\phi(x,y)$.

\fbox{\begin{minipage}{\dimexpr\textwidth-2\fboxsep-2\fboxrule\relax}
\paragraph{[1.3.1] Teorema} Sea $D$ un conjunto abierto y simplemente conexo y $M, N$ campos 
escalares definidos en $D$ de clase $C^1$. Bajo estas condiciones $ M(x,y) + N(x,y) y' = 0$ es una ecuación 
diferencial exacta en $D$ sii:
\begin{equation*}
   \frac{\partial M}{\partial y} = \frac{\partial N}{\partial x}, \forall (x,y) \in D \subseteq \mathbb{R}^2
\end{equation*}
\end{minipage}}

Como podemos ver, para saber si una ecuación es diferencial exacta solo basta checar: 
\begin{enumerate}[1., topsep=-2pt, itemsep=-2pt]
\item Que $D$ sea simplemente conexo y acotoado
\item $M, N \in C^1[D]$
\item La igualdad de las derivadas parciales de $M,N$ que están en el teorema
\end{enumerate}
Una vez que ya verificamos los tres pasos, la metodología para resolver estás ecuaciones es bastante sencilla. Como ya 
sabemos que la ecuación es diferencia exacata, entonces sabemos que $\exists \phi : \nabla \phi = (M, N)$. Por lo que 
\begin{equation*}
   \frac{\partial \phi}{\partial x} = M(x,y),  \quad \frac{\partial \phi}{\partial y} = N(x,y)
\end{equation*}
Si arbitrariamente integramos una derivada parcial, para este ejemplo usaré ${\partial \phi}/{\partial x}$, tendremos:
\begin{equation*}
   \phi(x,y) = \int M(x,y) dx + g(y)
\end{equation*}
Como se puede observar se tiene que encontrar el termino $g(y)$. Hay que tomar derivada parcial respecto a la otra variable
, en este caso $y$; tendremos lo siguiente
\begin{equation*}
   \frac{\partial \phi}{\partial y} = \frac{\partial}{\partial y} \left[ \int M(x,y) dx \right] + g'(y) = N(x,y)
\end{equation*}
Entonces, para encontrar $\phi(x,y)$ hay que despejar $g'(y)$ e integrar. Una vez teniendo $\phi(x,y)$ como vimos anteriormente, 
la solución implítica de la ecuación diferencial se verá: $\phi(x,y) = k, k \in \mathbb{R}$.
\subsubsection{Factor integrante}
Como vimos, si queremos resolver una ecuación diferencial de la forma $M(x,y) + N(x,y) y' = 0$, sólo 
basta checar que se cumplan las condiciones del teorema [1.3.1]. Sin embargo, ¿qué pasaría si la única 
condición que no se cumple es $\frac{\partial M}{\partial y} = \frac{\partial N}{\partial x}$?
La manera en la que podemos resolver este tipo de ecuaciones es agregando una función (conocida como el \textbf{factor integrante}) $\mu(x,y)$ tal que la ecuación
\begin{equation*}
   \mu(x,y) M(x,y) + \mu(x,y) N(x,y) y' = 0
\end{equation*}
sea diferencial exacta. 
Para encontrar dicho factor se tiene que resolver la siguiente ecuación:
\begin{equation*}
   \frac{\partial}{\partial y} \left[\mu (x,y) M(x,y) \right] = \frac{\partial}{\partial x}[\mu (x,y) N(x,y) ]
\end{equation*}
Una vez obtenido $\mu(x,y)$, definimos $\hat{M}(x,y) := \mu(x,y) M(x,y) $ y $\hat{N}(x,y) := \mu(x,y) N(x,y) $. Para luego resolver
la ecuación diferencial exacta: 
\begin{equation*}
   \hat{M}(x,y) + \hat{N}(x,y) y' = 0
\end{equation*}
Una forma de verificar que $\mu(x,y)$ es el factor integrante, es verificando que efectivamente la nueva ecuación es diferencial exacta.
\subsection{Variación de las constantes}
Sea la ecuación \textbf{[1.4.1]}:
\begin{equation*}
  P_1(x)y' + P_0(x)y = r(x)  
\end{equation*}
tal que $P_1, P_0 \text{ y } r$ son funciones continuas en un intervalo $D$ y $P_1(x) \neq 0$.

\fbox{\begin{minipage}{\dimexpr\textwidth-2\fboxsep-2\fboxrule\relax}
\paragraph{Definición} La  \textbf{ecuación homogénea asociada} a [1.4.1] es la ecuación de variables separables es 
$P_1(x)y' + P_0(x)y = 0$.
\end{minipage}}

La manera en la que vamos a resolver ecuaciones de la forma [1.4.1], será encontrando la solución general de su ecuación 
homogénea asociada: 
\begin{equation*}
   y(x) = \exp{\left(- \int \frac{P_0(x)}{P_1(x)} dx\right)} \cdot k, \quad k \in \mathbb{R}_{\neq 0}
\end{equation*}

El siguiente pasó será suponer que la variable $k$ es una función que toma valores el en dominio $D$. Así pues, la solución tendrá la siguiente forma:
\begin{equation*}
   y(x) = \exp{\left(- \int \frac{P_0(x)}{P_1(x)} dx\right)} \cdot k(x), \quad k : D \to \mathbb{R}
\end{equation*}

Ahora sustituimos la solución en la ecuación [1.4.1] : 
\begin{equation*}
   P_1(x) \frac{d }{dx}\left[ E(x) k(x) \right]+ P_0(x) E(x) k(x) = r(x)
\end{equation*}
Donde $ E(x) := \exp\left(- \int \frac{P_0(x)}{P_1(x)} dx\right)$, simplificando el término se llegará a: 
\begin{equation*}
   P_1(x)e^{- \int{\frac{P_0(x)}{P_1(x)}} dx} k'(x) = r(x)
\end{equation*}
Luego despejamos $k'(x)$, obtenemos $k(x)$ (sin olvidar la constante de integración) para finalmente sustituir $k(x)$ en la solución.



\textbf{Nota:} La intuición detrás del proceso de resolver este tipo de ecuaciones usando el método de la variación de las constantes,
parte de la idea de que estamos buscando el espacio nulo (de funciones diferenciables) asociado a la transformación $[P_1(x), P_2(x)]$ para luego
poder resolver la ecuación no homogénea.
\subsection{Ecuación diferencial de Bernoulli}
\fbox{\begin{minipage}{\dimexpr\textwidth-2\fboxsep-2\fboxrule\relax}
\paragraph{Definición} Se dice que una ecuación es diferencial de Bernoulli cuando:
\begin{equation*}
   y' + P(x)y = Q(x)y^{\alpha}, \quad Q,P : D \subseteq \mathbb{R} \to \mathbb{R}
\end{equation*}
\end{minipage}}

Se puede observar que si $\alpha \in \{0,1\}$ entonces la solución de dicha ecuación será bastante sencilla considerando 
que ya sabemos resolver una ecuación de variables separables (si $\alpha = 1$) y una ecuación lineal si ($\alpha = 0$). El 
caso de nuestro interés será cuando: $\alpha \notin \{0,1\}$.

Bajo dicho supuesto ($\alpha \notin \{0,1\}$), multipliquemos la ecuación diferencial de Bernoulli por $(1-\alpha)$ y $y^{-\alpha}$, se llegará a la siguiente ecuación:
\begin{equation*}
   (1 - \alpha) y' \cdot y^{-\alpha} + (1 - \alpha)P(x) \cdot y^{1-\alpha} = (1 - \alpha)Q(x)
\end{equation*}
Sea $z := y^{1-\alpha}$, entonces $z' = (1 - \alpha) y^{-\alpha} y'$, por lo que si ponemos la ecuación de arriba en términos de 
$z$ llegamremos a:
\begin{equation*}
   z' + z(1 -\alpha)P(x) = (1 -\alpha)Q(x)
\end{equation*}
Podemos observar que esta ecuación se puede resolver usando el método de variación de constantes. Resolvemos dicha ecuación 
para encontrar $z$, entonces para finalmente obtener $y$ de $y^{1-\alpha} = z$.
\subsection{Ecuación diferencial de Ricati}
\fbox{\begin{minipage}{\dimexpr\textwidth-2\fboxsep-2\fboxrule\relax}
\paragraph{Definición} Se dice que una ecuación es diferenical de Riccati cuando:
\begin{equation*}
   y' = P(x)y^2 +Q(x)y + R(x), \quad P,Q,R : D \subseteq \mathbb{R} \to \mathbb{R}
\end{equation*}
\end{minipage}}

Si una soución de la ecuación diferencial de Riccati es conocida $y_0$. Podemos definir la solución general como: 
\begin{equation*}
   y(x) = y_0(x) + \frac{1}{z(x)} \Longrightarrow y'(x) = y'_0 - \frac{z'(x)}{z^2(x)}
\end{equation*}
Ahora si sustituimos la expresión de la solución general en la ecuación de Riccati, tendremos: 
\begin{equation*}
   y'_0(x)\frac{z'(x)}{z^2(x)} = P(x) \left(y_0(x) + \frac{1}{z(x)}\right)^2 + Q(x) \left(y_0(x) + \frac{1}{z(x)}\right) + R(x)
\end{equation*}
Desarrollando dicha expresión se llegará a: 
\begin{equation*}
   z' + z(2P(x) y_0+Q(x)) = -P(x)
\end{equation*}
Podemos apreciar que dicha ecuación se puede resolver usando el método de variación de las constantes para 
encontrar $z$. Una vez teniendo $z$, lo sustituimos en la expresión de la solución general para la ecuación de Riccatti.
\section{Ecuaciones lineales de segundo orden}
Sea la siguiente una ecuación lineal de segundo orden:
\begin{equation*}
   P_2(x)y'' + P_1(x)y'+P_0(x)y = 0, \quad P_2, P_1, P_0 : D \subseteq \mathbb{R} \to \mathbb{R}
\end{equation*}
Siempre y cuando $P_2(x) \neq 0$. Ahora bien, no existe un método general para resolver este tipo de ecuaciones sólo hay 
para casos particulares.

\fbox{\begin{minipage}{\dimexpr\textwidth-2\fboxsep-2\fboxrule\relax}
\paragraph{Definición} El  \textbf{Wronskiano} de dos funciones $f, g$ e se define: 
\begin{equation*}
  W[f, g](x) = \det
   \begin{bmatrix}
      f(x) & g(x) \\ 
      f'(x) & g(x)
   \end{bmatrix}  = f(x)g'(x) + g(x)f'(x)
\end{equation*}
\end{minipage}}

\fbox{\begin{minipage}{\dimexpr\textwidth-2\fboxsep-2\fboxrule\relax}
\paragraph{[2.1] Teorema} Sean $y_1$ y $y_2$ soluciones de $P_2(x)y'' + P_1(x)y'+P_0(x)y = 0$, ${y_1, y_2}$ es 
un conjunto linealmente independiente si
\begin{equation*} 
   W[y_1, y_2](x) \neq 0, \text{ para algún } x_0 \in D
\end{equation*}
\end{minipage}}

\fbox{\begin{minipage}{\dimexpr\textwidth-2\fboxsep-2\fboxrule\relax}
\paragraph{[2.2] Teorema} El Wronskiano verifica la igualdad de Abet
\begin{equation*}
   W(x) = W(x_0)\exp \left( {\int_{x_0}^{x}\frac{P_1(t)}{P_2(t)}dt} \right),\quad  x_0 \in D
\end{equation*}
\end{minipage}}



\subsection{Solución particular de una ecuación homogénea}
\subsection{Ecuación homogénea con coeficientes constantes}

\section{Ecuaciones de Euler}

\end{document}